\section{syntax}

We can see the grammar of X language in table \ref{syntax}. It's written in BNF\cite{bnf}.

\begin{longtable}{rcl}
    \hline
    $ (\!\textit{Integer})\ n $ & $ \in $ & $ Int $\\
    $ (\!\textit{Bool})\ b $ & $ \in $ & $ \{\texttt{true}, \texttt{false}\} $\\

    $ passive\_operation $ & $::=$ & $ \texttt{int}/n(-;-) $\\
     && $ \texttt{bool}/b(-;-) $\\
     && $ \texttt{func}/n(v_1\dots v_n.e)$\\ 
     $ active\_operation $ & $::=$ & $int\_arith$\\
     && $ bool\_arith $\\ \hline
     \caption{Syntax}
     \label{syntax}
\end{longtable}

Describe the scope of variable binding informally...

And we can write a sample program like \ref{code:sample}.

\begin{listing}[H]
    \begin{minted}[frame=lines,baselinestretch=1,fontsize=\footnotesize]{text}
        &&(
            >(2, 1);
            _.||(!=(0, 1), false)
        )
    \end{minted}
	\caption{Sample Code}
    \label{code:sample}
\end{listing}

If you like, you can have 中文.