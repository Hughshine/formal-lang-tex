% * Syntax(scope) 
% * Simple type system(typing rule, scope rule)
% * Operational Semantics
%     * State 
%         Notation 
%     * Structural operational semantics
%         * use substitution to close a term
\documentclass[12pt,openright,bibliography=totoc]{article}
\usepackage[UTF8]{ctex}
% \usepackage{ctex}

\def\contentsname{Contents}
\def\tablename{Table}
\def\figurename{Figure}

\usepackage{ifxetex}
\ifxetex
  \usepackage[bookmarksnumbered]{hyperref}
\else
  \usepackage[unicode,bookmarksnumbered]{hyperref}
\fi

\usepackage{amssymb}
\usepackage[nottoc,numbib]{tocbibind}
%   根据需要选择 biblatex 宏包选项.
%\usepackage[maxnames=3,minnames=3,sorting=none,style=xxx]{biblatex}
\usepackage[maxnames=3,minnames=3,sorting=none]{biblatex}
% \usepackage{cite}
\hypersetup{colorlinks=true,
            pdfborder=0 0 1,
            citecolor=black,
            linkcolor=black}
%\usepackage{tikz}
\usepackage{amsmath}
\usepackage{booktabs}
\usepackage{graphicx}
\usepackage{multirow}
\usepackage{pgfplots}
\pgfplotsset{compat=1.13}
\usepackage[utf8]{inputenc}
\usepackage{graphicx} 
\usepackage{auto-pst-pdf}
\usepackage[pdf]{graphviz}
\usepackage[section]{placeins}

% \usepackage{subfigure}
\usepackage{subcaption}
\usepackage{listings}
\usepackage{proof}
\usepackage{mathpartir}
\usepackage{stmaryrd}
\usepackage{pgfplots}
\usepackage{tikz}
\usepackage{setspace}
\usepackage{color,xcolor}
% \usepackage{amsmath}
\usepackage{amssymb}
\usepackage{stmaryrd}
\let\stmaryrdLightning\lightning
\usepackage{dblfloatfix}
\usepackage{lastpage}

\usepackage{algorithm}
\usepackage{algpseudocode}
\usepackage{cases}
\usepackage{multirow}
\usepackage{tabularx}
\usepackage{fancyvrb}
\usepackage{xparse}
\usepackage{alltt}
\usepackage{longtable}
\usepackage[newfloat]{minted}
\usepackage{fancyhdr}

% \usepackage{multicol,longtable,caption,tabu}
% \usepackage{multicap}
% \usepackage{ksafe-lin-proof}
\usepackage{mathtools}
\usepackage{tabularx}  
\usepackage{tikz}
\usepackage{makecell}
\usepackage{fancyvrb}
\usepackage{enumitem}

\usetikzlibrary{trees}
\usetikzlibrary{graphs}
\usetikzlibrary{arrows}

\tikzset{
      treenode/.style = {align=center, inner sep=0pt, text centered,
        font=\sffamily},
      arn_n/.style = {treenode, circle,black, draw=black, text width=1.3em},% 
    level distance = 1cm,
    level 1/.style={sibling distance=2cm},
    level 2/.style={sibling distance=1cm},
    level 3/.style={sibling distance=1cm}
}

% 设置页边距,set margins
\usepackage[left = 15mm, right = 15mm, top = 20mm, bottom= 20mm]{geometry}
\geometry{a4paper,left=2.3cm,right=2.3cm,top=2.7cm,bottom=2.7cm}
\setitemize[1]{itemsep=0pt,partopsep=0pt,parsep=\parskip,topsep=5pt}

\definecolor{corange}{HTML}{FF9666}
\definecolor{lblue}{HTML}{97FFFF}
\definecolor{cblue}{HTML}{AB82FF}
\definecolor{cred}{HTML}{FF99A1}
\definecolor{cgreen}{HTML}{7ACCBE}
\definecolor{red1}{RGB}{254,67,101}
\definecolor{red2}{RGB}{252,157,154}
\definecolor{red3}{RGB}{249,205,173}
\definecolor{grey1}{RGB}{200,200,169}
\definecolor{grey2}{RGB}{131,175,155}
\definecolor{color1}{HTML}{0099CC}
\definecolor{color2}{HTML}{F9B46F}
\definecolor{cblue1}{HTML}{0074D9}
\definecolor{lblue1}{HTML}{99CCFF}
\definecolor{corange1}{HTML}{FFDDCE}
\definecolor{corange2}{HTML}{FF9900}
\definecolor{corange3}{HTML}{FF9966}
\definecolor{corange4}{HTML}{FF6600}
\definecolor{cred2}{HTML}{FF6666}
\usetikzlibrary{patterns}
\addbibresource{main.bib}
\DeclareBibliographyCategory{cited}
\AtEveryCitekey{\addtocategory{cited}{\thefield{entrykey}}}

\includeonly{
syntax, 
typesystem,
state,
semantics,
assertionlang,
references
}

\pagestyle{fancy}
\lhead{Techinical Setting for \textbf{X} Language}
\rhead{\rightmark}
\cfoot{\thepage\ of \pageref{LastPage}}%当前页 of 总页数
\renewcommand{\headrulewidth}{0.4pt}%改为0pt即可去掉页眉下面的横线
\renewcommand{\footrulewidth}{0.4pt}%改为0pt即可去掉页脚上面的横线

% \newtheorem{Theorem}{\hskip 2em 定理}[section]
% \newtheorem{Lemma}[Theorem]{\hskip 2em 引理}
% \newtheorem{Corollary}[Theorem]{\hskip 2em 推论}
% \newtheorem{Proposition}[Theorem]{\hskip 2em 命题}
% \newtheorem{Definition}[Theorem]{\hskip 2em 定义}
% \newtheorem{Example}[Theorem]{\hskip 2em 例}
% \newcommand{\upcite}[1]{\textsuperscript{\textsuperscript{\cite{#1}}}}
% \renewcommand{\algorithmicrequire}{\textbf{输入:}}
% \renewcommand{\algorithmicensure}{\textbf{输出:}}
% \algnewcommand{\LeftComment}[1]{\Statex \(\triangleright\) #1}
% \floatname{algorithm}{算法}

\newcommand{\abs}[1]{\lvert#1\rvert}
\newcommand{\norm}[1]{\lVert#1\rVert}
\newcommand{\iv}{\texttt{i}_v}
\newcommand{\ia}{\texttt{i}_a}
\newcommand{\iL}{\texttt{i}_L}
\newcommand{\kspace}{\vspace{0.5cm}}
\algnewcommand\algorithmicmatch{\textbf{match}}
\algnewcommand\algorithmicwith{\textbf{with}}
\algnewcommand\algorithmiccase{\textbf{case}}
\algnewcommand\algorithmicassert{\texttt{assert}}
\algnewcommand\Assert[1]{\State \algorithmicassert(#1)}%
% New "environments"
\algdef{SE}[MATCH]{Match}{EndMatch}[1]{\algorithmicmatch\ #1\ \algorithmicwith}{\algorithmicend\ \algorithmicmatch}%
\algdef{SE}[CASE]{Case}{EndCase}[1]{\algorithmiccase\ #1}{\algorithmicend\ \algorithmiccase}%
\algtext*{EndSwitch}%
\algtext*{EndCase}%

\title{Techinical Setting for \textbf{X} Language}
\author{Apprentice}
\date{Jan. 2000}

\begin{document}

\maketitle

\tableofcontents

\newpage

\section{syntax}

We can see the grammar of X language in table \ref{syntax}. It's written in BNF\cite{bnf}.

\begin{longtable}{rcl}
    \hline
    $ (\!\textit{Integer})\ n $ & $ \in $ & $ Int $\\
    $ (\!\textit{Bool})\ b $ & $ \in $ & $ \{\texttt{true}, \texttt{false}\} $\\

    $ passive\_operation $ & $::=$ & $ \texttt{int}/n(-;-) $\\
     && $ \texttt{bool}/b(-;-) $\\
     && $ \texttt{func}/n(v_1\dots v_n.e)$\\ 
     $ active\_operation $ & $::=$ & $int\_arith$\\
     && $ bool\_arith $\\ \hline
     \caption{Syntax}
     \label{syntax}
\end{longtable}

Describe the scope of variable binding informally...

And we can write a sample program like \ref{code:sample}.

\begin{listing}[H]
    \begin{minted}[frame=lines,baselinestretch=1,fontsize=\footnotesize]{text}
        &&(
            >(2, 1);
            _.||(!=(0, 1), false)
        )
    \end{minted}
	\caption{Sample Code}
    \label{code:sample}
\end{listing}

If you like, you can have 中文.
\section{type system}

We have a sample type system, in table \ref{tb:type-system}.

\begin{table}[h]
    \centering
    \[
    \arraycolsep=1em
    \def\arraystretch{2.5}
    \begin{array}{c}
    % \hline
            \begin{array}{ccc}
            \infer
                {\vdash \textit{Int} : int}
                {\;}

            &
            \infer
                {\vdash \textit{Real} : real}
                {\;}
            & 
            \infer 
                {\vdash \texttt{(}e\texttt{)}:\tau}
                {\vdash e:\tau}
        \end{array}
        \\
        \begin{array}{c}
            \infer
                {\vdash e_1\;op\;e_2 : \tau}
                {\vdash e_1:\tau\quad \vdash e_2:\tau}
        \end{array}
    \end{array}
    \]
    \caption{Sample Type System}
    \label{tb:type-system}
\end{table}
\section{state}

\begin{longtable}{rcl rcl rcl}
    \hline			
    $ (\!\textit{Value})\ w, r $ & $ \in $ & $\mathbb{O}_{\checkmark}$\\  
    $ (\!\textit{LVMap})\ s $ & $ \in $ & $ Nat \rightharpoonup Term$\\
    $ (\!\textit{RefMap})\ h $ & $ \in $ & $ Nat \rightharpoonup Term$\\
    $ (\!\textit{BindState})\ \sigma $ & $ ::= $ & $ (s, h) $\\
    
    $  (\!\textit{Tag-Block})\ T_b $ & $ \in $ & $\{\texttt{ok}, \texttt{break}, \texttt{continue}\}$\\  
    $ (\!\textit{BlockState})\ \gamma $ & $ ::= $ & $ (T_b, \iL, r)$\\
    
    $ (\!\textit{State})\ \mathcal{P} $ & $ ::= $ & $ (\sigma,\gamma, \delta) $\\
    \hline
    \caption{State Design}
    \label{tb:formal-state}\\
\end{longtable}

\begin{longtable}{rcl rcl rcl}
    \hline			
    $ (\!\textit{VarAccess})\ \iv \mapsto_s t $ & $ ::= $ & $ s(\iv) = t$\\  
    $ (\!\textit{ReferenceAccess})\ \ia \mapsto_h v $ & $ ::= $ & $ h(\ia) = v$\\ 
    $ (\!\textit{MapUpdate})\ s\{k \rightsquigarrow v\} $ & $ ::= $ & $\lambda z.\left\{\def\arraystretch{1.2}\begin{tabular}{@{}l@{\quad}l@{}}$s(z),$ &\text{if}\;$z\not=k;$ \\$v$, & \text{if}\;$z=k.$\end{tabular}\right.$ \\
    \hline
    \caption{Notations}
    \label{tb:notations}\\	
\end{longtable}
% \include{assertionlang}
\section{semantics}

\begin{table}[h]
	\centering
	\begin{tabular}{c}
    	$
        	\inferrule[Eval-Loc-Var] %  evaluation finishes
        	{ \iv \mapsto_s t}
        	{ (\iv, (s, h), -, -, -) \longrightarrow  (t, (s, h), -, -, -)}
        $  \kspace \\   	

    	$
    	\inferrule[Bind-Var]
    	{ }
    	{(\texttt{bind}\;x = t\;\texttt{in}\;e, \sigma, (\vec{x}, \vec{t^x}, \vec{a}, \vec{t^a}),  -, -) \longrightarrow (e, \sigma, (\vec{x}::x, \vec{t^x}::t, \vec{a}, \vec{t^a}), -, -)}
    	$ \kspace \\
    	
    	$
    	\inferrule[Bind-Name]
    	{ }
    	{(\texttt{new}\;x = t\;\texttt{in}\;e, \sigma, (\vec{x}, \vec{t^x}, \vec{a}, \vec{t^a}),  -, -) \longrightarrow (e, \sigma, (\vec{x}, \vec{t^x}, \vec{a}::x, \vec{t^a}::t),  -, -)}
    	$     	\kspace		\\

    	$
    	\inferrule[Bind-OK]  % v_i不会被替换,但可能用于替换
    	{ k_x = |\vec{x}|\quad k_a = |\vec{a}|\quad k_x > 0 \vee k_a > 0\\
        \quad \forall i \in \{1, \dots, k_x\}\; {\iv}_i = new\_loc_v() \\
        \quad \forall j \in \{1, \dots, k_a\}\; {\ia}_j = new\_loc_a() \\
    	\quad \forall i \in \{1,\dots, k_x\}\;t_i^{x\prime} = t_i^x[{\iv}_1/x_1]\dots[{\iv}_{k_x}/x_i][{\ia}_1/a_1]\dots[{\ia}_{k_a}/a_{k_a}] \\
    	\quad \forall j \in \{1,\dots, k_a\}\;t_j^{a\prime} = t_j^a[{\iv}_1/x_1]\dots[{\iv}_{k_x}/x_{k_x}][{\ia}_1/a_1]\dots[{\ia}_{k_a}/a_{k_a}] \\
    	\quad t' = t[{\iv}_1/x_i]\dots[{\iv}_{k_x}/x_i][{\ia}_1/a_1]\dots[{\ia}_{k_a}/a_{k_a}]\\
    	\quad s' = s\{{\iv}_1 \rightsquigarrow {t_1^{x\prime}},\dots, {\iv}_{k_x} \rightsquigarrow {{t_{k_x}^{x\prime}}}\} \\
    	\quad h' = h\{{\ia}_1\rightsquigarrow {t_1^{a\prime}}, \dots, {\ia}_{k_a}\rightsquigarrow {{{t}_{k_a}^{a\prime}}}\} \\
    	}
    	{(t, (s, h), (\vec{x}, \vec{t^x}, \vec{a}, \vec{t^a}), -, -) \longrightarrow (t', (s', h'), -, -, -)}
    	$ \kspace \\   
            
        $
    	\inferrule[Eval-Thunk-Apply] % evaluate a var bind with a term
    	{n = \abs{\vec{x}}\quad \abs{\vec{x}} = \abs{\vec{r}}\quad \forall\;i\in\{1,\dots,n\}\quad{\iv}_i = new\_loc_v()}
    	{ (\vec{x}(\vec{r}).e, (s, h), -, -, -) \longrightarrow\\ (e[{\iv}_1/x_1]\dots[{\iv}_n/x_n], (s\{{\iv}_1\rightsquigarrow r_1,\dots,{\iv}_n\rightsquigarrow r_n\}, h), -, -, -)}
    	$ \\
	\end{tabular}
    \caption{Operational Semantics for Spartan}
    \label{tb:formal-semantics-bind}
\end{table}
\def\bibrangedash{$\sim$}
\printbibliography[
    heading=bibintoc,
    title={References},
    category=cited
]

\end{document}